\section{Appendix: Scriptimplementierung}

\subsection{tsch\_scenario.cc}

\begin{lstlisting}[frame=single]
#include <ns3/log.h>
#include <ns3/core-module.h>
#include <ns3/network-module.h>
#include <ns3/lr-wpan-module.h>
#include <ns3/simulator.h>
#include <ns3/single-model-spectrum-channel.h>
#include <ns3/packet.h>
#include <ns3/mobility-module.h>
#include <ns3/spectrum-helper.h>
#include <string>

#include <iostream>

using namespace ns3;

/////////////////////////////////
// Configuration
/////////////////////////////////
bool verbose = true;       // enable logging
uint32_t numberOfFFDs = 7;      // FFDs
bool brdcst_as_join = true;
bool collision = true;

int pktsize = 91;           //size of packets, in bytes
double duration = 1;       //simulation total duration, in seconds
double starttopancoord = 0.02;        // Starttime transmitting FFDs
double interval = 0.2;
double starttoffds = 0.08;

void
LogComponents(bool phy, bool csmaca, bool diverse)
{
  LogComponentEnableAll (LOG_PREFIX_TIME);
  LogComponentEnableAll (LOG_PREFIX_FUNC);
  LogComponentEnable ("LrWpanTschMac", LOG_LEVEL_ALL);
  LogComponentEnable ("LrWpanTschNetDevice", LOG_LEVEL_ALL);
  if (phy)
  {
    LogComponentEnable ("LrWpanPhy", LOG_LEVEL_ALL);
    LogComponentEnable ("LrWpanSpectrumSignalParameters", LOG_LEVEL_ALL);
    LogComponentEnable ("LrWpanSpectrumValueHelper", LOG_LEVEL_ALL);
  }
  if (csmaca)
  {
    LogComponentEnable ("LrWpanCsmaCa", LOG_LEVEL_ALL);
  }
  if (diverse)
  {
    LogComponentEnable ("LrWpanErrorModel", LOG_LEVEL_ALL);
    LogComponentEnable ("LrWpanInterferenceHelper", LOG_LEVEL_ALL);
  }

}

void
EnableTsch(LrWpanTschHelper* lrWpanHelper,NetDeviceContainer& netdev)
{
  lrWpanHelper->EnableTsch(netdev,0,duration);
}

int main (int argc, char** argv)
{

  CommandLine cmd;
  cmd.AddValue ("verbose", "Print trace information if true", verbose);
  cmd.Parse (argc, argv);
  GlobalValue::Bind ("ChecksumEnabled", BooleanValue (true));

  // ----------------------------
  // Nodes
  NodeContainer ffds;
  NodeContainer panCoord;
  ffds.Create (numberOfFFDs);
  panCoord.Create (1);
  NodeContainer lrwpanNodes(panCoord, ffds);

  /////////////////////////////////
  // Mobility
  /////////////////////////////////

  MobilityHelper mobility;
  mobility.SetMobilityModel ("ns3::ConstantPositionMobilityModel");
  mobility.SetPositionAllocator ("ns3::GridPositionAllocator",
                                 "GridWidth", UintegerValue(4),
                                 "MinX", DoubleValue (0.0),
                                 "MinY", DoubleValue (0.0),
                                 "DeltaX", DoubleValue (5),
                                 "DeltaY", DoubleValue (5),
                                 "LayoutType", StringValue ("RowFirst"));
  mobility.Install (lrwpanNodes);

  /////////////////////////////////
  // Channel
  /////////////////////////////////
  SpectrumChannelHelper channelHelper = SpectrumChannelHelper::Default ();
  channelHelper.SetChannel ("ns3::MultiModelSpectrumChannel");
  Ptr<SpectrumChannel> channel = channelHelper.Create ();

  /////////////////////////////////
  // Configure lrwpan nodes
  /////////////////////////////////
  LrWpanTschHelper lrWpanHelper(channel,numberOfFFDs+1,false,true);

  // Add and install the LrWpanTschNetDevice for each node
  NetDeviceContainer netdev = lrWpanHelper.Install (lrwpanNodes);

  // AssociateToPan
  lrWpanHelper.AssociateToPan(netdev,123);

  // Slotframes
  lrWpanHelper.ConfigureSlotframeAllToPan(netdev,0,false,true); // slotframes = 8

  // start with TSCH
  Simulator::Schedule(Seconds(0),&EnableTsch,&lrWpanHelper,netdev);

  // Packets from panCoord
  Ptr<Node> panCoordNode = panCoord.Get (0);
  Ptr<NetDevice> panCoordNetDevice = panCoordNode->GetDevice (0);
  Address address_panCoord = panCoordNetDevice->GetAddress ();

  // panCoord
  // Broadcast Address
  Mac16Address brdcst ("ff:ff");
  // Advertisment Link
  lrWpanHelper.GenerateTraffic (panCoordNetDevice, brdcst, pktsize, 0.0, duration, interval);

  // Broadcast Link
  if (brdcst_as_join)
  {
    // Send File from FFD to pancoord during Brdcst link
    Ptr<Node> brdcst_ffd_one = ffds.Get (0); // 00:01
    Ptr<NetDevice> brdcst_ffd_one_netdev = brdcst_ffd_one->GetDevice (0);
    Ptr<Node> brdcst_ffd_two = ffds.Get (1); // 00:02
    Ptr<NetDevice> brdcst_ffd_two_netdev = brdcst_ffd_one->GetDevice (0);

    if (collision)
    {
      // send to pancoord
      lrWpanHelper.GenerateTraffic (brdcst_ffd_one_netdev, address_panCoord, pktsize, 0.1, duration, interval);
      lrWpanHelper.GenerateTraffic (brdcst_ffd_two_netdev, address_panCoord, pktsize, 0.1, duration, interval);
      // send brdcst
      lrWpanHelper.GenerateTraffic (brdcst_ffd_one_netdev, brdcst, pktsize, 0.1, duration, interval);
      lrWpanHelper.GenerateTraffic (brdcst_ffd_two_netdev, brdcst, pktsize, 0.1, duration, interval);
    }
    else
    {
      lrWpanHelper.GenerateTraffic (brdcst_ffd_one_netdev, address_panCoord, pktsize, 0.1, duration, interval);
    }
  }
  else
  {
    //Address brdcst = panCoordNetDevice->GetBroadcast (); // ff:ff
    lrWpanHelper.GenerateTraffic (panCoordNetDevice, brdcst, pktsize, 0.1, duration, interval);
  }

  // FFDs
  for (NodeContainer::Iterator i = ffds.Begin (); i != ffds.End (); i++)
  {
    Ptr<Node> node = *i;
    Ptr<NetDevice> device = node->GetDevice (0);
    lrWpanHelper.GenerateTraffic (device, address_panCoord, pktsize, starttopancoord, duration, interval);
    starttopancoord += 0.01; // starttopancoord = 0.02
  }

  //Enable PCAP and Ascii Tracing
  AsciiTraceHelper ascii;
  Ptr<OutputStreamWrapper> stream = ascii.CreateFileStream ("tsch_scenario.tr");
  lrWpanHelper.EnablePcapAll (std::string ("tsch_scenario"), true);
  lrWpanHelper.EnablePcap (std::string ("tsch_scenario"), panCoordNetDevice, true);
  lrWpanHelper.EnableAsciiAll (stream);
  if (verbose)
    {
      //lrWpanHelper.EnableLogComponents ();
      LogComponents(false, false, false);
    }

  // ------------------------------

  Simulator::Run ();
  Simulator::Destroy ();
  return 0;
}

\end{lstlisting}

\subsection{tsch\_scenario\_collision.cc}
\begin{lstlisting}[frame=single]
#include <ns3/log.h>
#include <ns3/core-module.h>
#include <ns3/network-module.h>
#include <ns3/lr-wpan-module.h>
#include <ns3/simulator.h>
#include <ns3/single-model-spectrum-channel.h>
#include <ns3/packet.h>
#include <ns3/mobility-module.h>
#include <ns3/spectrum-helper.h>
#include <string>

#include <iostream>

using namespace ns3;

/////////////////////////////////
// Configuration
/////////////////////////////////
bool verbose = true;       // enable logging
uint32_t numberOfFFDs = 2;      // FFDs

int pktsize = 91;           //size of packets, in bytes
double duration = 1;       //simulation total duration, in seconds
double starttopancoord = 0.01;        // Starttime transmitting FFDs
double interval = 0.1;
double starttoffds = 0.08;


void
LogComponents(bool phy, bool csmaca, bool diverse)
{
  LogComponentEnableAll (LOG_PREFIX_TIME);
  LogComponentEnableAll (LOG_PREFIX_FUNC);
  LogComponentEnable ("LrWpanTschMac", LOG_LEVEL_ALL);
  LogComponentEnable ("LrWpanTschNetDevice", LOG_LEVEL_ALL);
  if (phy)
  {
    LogComponentEnable ("LrWpanPhy", LOG_LEVEL_ALL);
    LogComponentEnable ("LrWpanSpectrumSignalParameters", LOG_LEVEL_ALL);
    LogComponentEnable ("LrWpanSpectrumValueHelper", LOG_LEVEL_ALL);
  }
  if (csmaca)
  {
    LogComponentEnable ("LrWpanCsmaCa", LOG_LEVEL_ALL);
  }
  if (diverse)
  {
    LogComponentEnable ("LrWpanErrorModel", LOG_LEVEL_ALL);
    LogComponentEnable ("LrWpanInterferenceHelper", LOG_LEVEL_ALL);
  }

}

void
EnableTsch(LrWpanTschHelper* lrWpanHelper,NetDeviceContainer& netdev)
{
  lrWpanHelper->EnableTsch(netdev,0,duration);
}

int main (int argc, char** argv)
{

  CommandLine cmd;
  cmd.AddValue ("verbose", "Print trace information if true", verbose);
  cmd.Parse (argc, argv);
  GlobalValue::Bind ("ChecksumEnabled", BooleanValue (true));

  // ----------------------------
  // Nodes
  NodeContainer ffds;
  NodeContainer panCoord;
  ffds.Create (numberOfFFDs);
  panCoord.Create (1);
  NodeContainer lrwpanNodes(panCoord, ffds);

  /////////////////////////////////
  // Mobility
  /////////////////////////////////

  MobilityHelper mobility;
  mobility.SetMobilityModel ("ns3::ConstantPositionMobilityModel");
  mobility.SetPositionAllocator ("ns3::GridPositionAllocator",
                                 "GridWidth", UintegerValue(4),
                                 "MinX", DoubleValue (0.0),
                                 "MinY", DoubleValue (0.0),
                                 "DeltaX", DoubleValue (5),
                                 "DeltaY", DoubleValue (5),
                                 "LayoutType", StringValue ("RowFirst"));
  mobility.Install (lrwpanNodes);

  /////////////////////////////////
  // Channel
  /////////////////////////////////
  SpectrumChannelHelper channelHelper = SpectrumChannelHelper::Default ();
  channelHelper.SetChannel ("ns3::MultiModelSpectrumChannel");
  Ptr<SpectrumChannel> channel = channelHelper.Create ();

  /////////////////////////////////
  // Configure lrwpan nodes
  /////////////////////////////////
  LrWpanTschHelper lrWpanHelper(channel,numberOfFFDs+1,false,true);


  // Add and install the LrWpanTschNetDevice for each node
  NetDeviceContainer netdev = lrWpanHelper.Install (lrwpanNodes);

  // ------------------------------;

  // AssociateToPan
  lrWpanHelper.AssociateToPan(netdev,123);

  // Slotframes
  int size = 20;
  lrWpanHelper.AddSlotframe(netdev, 0, size);

  //Add links
  AddLinkParams alparams;
  alparams.slotframeHandle = 0;
  alparams.channelOffset = 0;

  uint16_t c = 1;

  for (u_int32_t i = 0; i < netdev.GetN ()-1; i++,c++)
    {
      alparams.linkHandle = c;
      alparams.timeslot = 1;
      lrWpanHelper.AddLink(netdev,i+1,0,alparams,true);
    }

  // start with TSCH
  Simulator::Schedule(Seconds(0),&EnableTsch,&lrWpanHelper,netdev);

  // Packets from panCoord
  Ptr<Node> panCoordNode = panCoord.Get (0);
  Ptr<NetDevice> panCoordNetDevice = panCoordNode->GetDevice (0);
  Address address_panCoord = panCoordNetDevice->GetAddress ();
  std::cout << address_panCoord << " ";

  // Broadcast Address
  Mac16Address mac16_brdcst ("ff:ff");

  // FFDs
  // Send File from FFD to pancoord during Brdcst link
  Ptr<Node> ffd_one = ffds.Get (0); // 00:01
  Ptr<NetDevice> ffd_one_netdev = ffd_one->GetDevice (0);
  Ptr<Node> ffd_two = ffds.Get (1); // 00:02
  Ptr<NetDevice> ffd_two_netdev = ffd_two->GetDevice (0);

  // send to pancoord
  lrWpanHelper.GenerateTraffic (ffd_one_netdev, address_panCoord, pktsize, 0.1, duration, interval);
  lrWpanHelper.GenerateTraffic (ffd_two_netdev, address_panCoord, pktsize, 0.1, duration, interval);

  //Enable PCAP and Ascii Tracing
  AsciiTraceHelper ascii;
  Ptr<OutputStreamWrapper> stream = ascii.CreateFileStream ("tsch_scenario_collision.tr");
  lrWpanHelper.EnablePcapAll (std::string ("tsch_scenario_collision"), true);
  lrWpanHelper.EnablePcap (std::string ("tsch_scenario_collision"), panCoordNetDevice, true);
  lrWpanHelper.EnableAsciiAll (stream);
  if (verbose)
    {
      //lrWpanHelper.EnableLogComponents ();
      // bool phy, bool csmaca, bool diverse
      LogComponents(true, true, true);
    }

  // ------------------------------
  Simulator::Run ();
  Simulator::Destroy ();
  return 0;
}
\end{lstlisting}
