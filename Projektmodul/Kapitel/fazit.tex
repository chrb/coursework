\section{Fazit und weiterf"uhrende Arbeiten} \label{Kap5-6}


\subsection{Grober Entwicklungsplan}

Anhand des Implementierungsstands m"ussen einige Entwicklungsarbeiten vorgenommen
werden, damit der Network Formation Process durchgef"uehrt werden kann. Nachfolgend
wird nun im Groben beschrieben was ein solcher Entwicklungsplan beinhaltet.

\begin{enumerate}
  \item Anpassen bzw. Neuentwicklung der Helpermethoden zur Erstellung
  von beliebigen Schedules mit beliebigen Variablenwerten f"ur Slotframes, Timeslots,
  ChannelOffset, SlotframeHandle mit den Linktypen Link, Advertisment Link und
  Broadcast Link.
  \item Analyse und Anpassung der aktuellen Grobstruktur der Information Elements
  \item Einbettung der IEs in die Module zur aktiven Anwendung
  \item Entwicklung von Helper Methoden zum Senden und Empfangen von IEs
  \item Helper Methoden und Einbettung der konkreten IEs innerhalb des Network
  Formation Process, wie in \ref{sec:Aufbau_Enhanced_Beacons} beschrieben
  \begin{itemize}
    \item Header IE Header
    \item Payload IE Header
    \item MLME-SubIE TSCH Synchronization
    \item MLME-SubIE TSCH Timeslot
    \item MLME-SubIE Channel Hopping
    \item MLME-SubIE TSCH Slotframe and Link
  \end{itemize}
  \item Beispiel zur Anwendung der IEs
  \item Implementierung des Advertisment Modes zum Aussenden der Enhanced Beacons
  \item Anpassen des "'Bootstrapping"'-Algorithmus anhand der neuen Implementierung.
\end{enumerate}
