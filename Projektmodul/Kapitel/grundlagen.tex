\section{Grundlagen} \label{grundlagen}

\subsection{IEEE 802.15.4 Low-Rate Wireless Personal Area Networks} \label{Kap5-2}

Der Standard IEEE 802.15.4 auch Low-Rate Wireless Personal Area Networks
LR-WPANs, beschreibt drahtlose Kommunikation f"ur Sensornetzwerke mit g"unstiger,
haupts"achlich batteriebetriebener Kommunikationshardware welche geringe Datenraten
erm"oglicht. Der Standard mitsamt seinem beschriebenen Protokoll bildet die
technologische Basis f"ur Anwendungen im Internet der Dinge IoT.

Die Grundversion offenbarte einige technische L"ucken und Nachteile im Vergleich
zu anderen Vergleichbaren Technologien (WLAN), weshalb mit dem \textit{Amendment A: MAC sublayer}
zus"atzliche Eigenschaften und Funktionen hinzugef"ugt wurden. Einige der wichtigsten
Probleme waren dabei,

\begin{itemize}
 \item keine garantierte Bandbreite ausserhalb von Guaranteed Time Slot GTS.
 \item keine Latenzbeschr"ankung.
 \item kein Schutz gegen Interferenzen und Fading.
 \item keine Technologie um die Frequenz des Kanals zu wechseln.
 \item einen unzuverl"assigen MAC Sublayer durch die Verwendung von CSMA/CA
      Die Wahrscheinlichkeit von Kollisionen "uber den Kanal steigt mit
      der Anzahl der Knoten durch erh"ohte Contention Time.
\end{itemize}

% ----------------------------------------------------------------------------

\subsection{IEEE 802.15.4e Amendment A: MAC sublayer} \label{Kap5-3}

Um die Probleme von 802.15.4 zu beheben wurde die 802.15.4e Working Group geschaffen,
welche das existierende MAC Protokoll verbessern sollte. Dadurch wurde
der 802.15.4 Standard um zwei Kategorien an MAC Verbesserungen erweitert.
Einmal den MAC Behavior Modes, welche spezifische Anwendungsbereiche verbessern,
sowie General Functional Enhancements f"ur generelle Verbesserungen. Bei der Entwicklung
wurden dabei viele Ideen aus bereits existierenden Standards wie WirelessHART und
ISA 100.11.a "ubernommen.

\begin{description}

\item[General Functional Enhancements] \hfill \\
  \begin{description}
    \item[Information Elements IE] sind bereits seit der Grundversion implementiert,
    "ubernehmen aber zus"atzliche Aufgaben im Amendment A.
    \item[Enhanced Beacons EB] bilden eine Erweiterung zu Beacon Frames mit gr"osserer
    Flexibili"at. Werden als anwendungsbezogene Beacons via IEs im TSCH Mode verwendet.

    \item[Low Energy LE] Austausch der Eigenschaften niedrige Latenz durch
    niedriger Energieverbrauch, wodurch das Ger"at immer erreichbar erscheint
%      \item[Multipurpose Frame] addressiert mehrere MAC Operations
%      \item[MAC Performance Metrics] stellen einen Mechanismus bereit um die
%      Kanalqualit"at zu ermitteln, wird f"ur das Networking sowie h"ohere Schichten
%      (6LoWPAN, RPL) ben"otigt.
%      \item[Fast Assocation FastA] wird bei zeit-kritischen Systemen angewendet.
  \end{description}

  \item[Behavior Modes] \hfill \\

    \begin{description}
      \item[Timeslotted Channel Hopping TSCH] f"ur die Verbesserung von automatisierten
      Anwendungsbereichen.
      \item[Deterministic and Synchronous Multi-Channel Extension DSME] zur Unterst"utzung
      von industriellen und kommerziellen Anwendungen, welche strikte Vorgaben
      an Zeit und Verf"ugbarkeit stellen.
    \end{description}
\end{description}

% ----------------------------------------------------------------------------

\subsection{Networksimulator ns-3} \label{Kap5-5}

Im Laufe der Projekts wird der Netzwerksimulator ns-3 verwendet, dabei
kann die Grobbeschreibung direkt von dem Internetauftritt unter
\href{http://www.nsam.org}{http://www.nsam.org} entnommen werden.

\begin{alltt}
ns-3 is a discrete-event network simulator for Internet systems,
targeted primarily for research and educational use.
ns-3 is free software, licensed under the GNU GPLv2 license,
and is publicly available for research, development, and use.
\end{alltt}

ns-3 ist konzipiert als eine Sammlung von Software-Bibliotheken, welche zusammen
ein System bilden. Diese einzelnen Bibliotheken werden als Module bezeichnet
und implementieren jeweils einen eigenen Standard, welche auf den Kernmodulen
aufbauen. Dadurch k"onnen Benutzerprogramme durch Verwendung dieser Module
aufgebaut werden.

Wir gehen im weiteren davon aus, dass der Leser mit ns-3 vertraut ist,
falls nicht, kann das mit Hilfe des \href{https://www.nsnam.org/ns-3-24/documentation/}{Tutorial}
nachgeholt werden.

%----------------------------------------------
\subsubsection{LR-WPAN Modul}

Im Verlauf dieser Arbeit wurde haupts"achlich das LR-WPAN Modul angewendet, welches
in der offiziellen ns-3 Version (NS-3.24) den IEEE 802.15.4 Standard mitsamt
dessen Funktionalit"aten und Hilfsanwendungen implementiert.

Der Aufbau des Modules erfolgt dabei nach ns-3 Standard und entspricht dabei der
Ordnerhierarchie \textit{models} mit den direkten Implementierungen des Standards,
\textit{helper} wichtigen und n"utzlichen Helperklassen und Methoden, zur einfachen
Nutzung des Standards sowie den selbsterkl"arenden Ordnern \textit{examples} und
\textit{test}.

F"ur eine genaue Beschreibung und Informationen zur Entwicklung des LR-WPAN Modules,
kann auf die Arbeit von Nikica Krezic-Luger\cite{bachelorarbeit} verwiesen werden.

Zus"atzlich zum offiziellen LR-WPAN Modul, wird nun das Entwicklerrepository ns-3-dev-tsch
der Entwicklungsgruppe \textit{RICH} von EIT Digital betrachtet. Dieses Repository
besteht aus einem vollst"andigen ns-3 Build mitsamt den Erweiterungen im LR-WPAN
Modul um die TSCH Mode Funktionalit"aten. Dieser Ansatz zur Erweiterung von ns-3
wurde im Januar 2015 zur Codereview freigegeben und kann in Github unter
\href{https://github.com/EIT-ICT-RICH/ns-3-dev-TSCH}{ns-3-dev-TSCH}
gefunden werden.

Zus"atzlich zum Code hat die Entwicklergruppe eine
\href{http://mailman.isi.edu/pipermail/ns-developers/2015-January/012459.html}{Verlautbarung}
zum Commitstand abgegeben, die folgende Funktionen im Detail beeinhaltet:

\begin{itemize}
  \item Wechsel von Standard 802.15.4 auf den TSCH Mode, w"ahrend der Kommunikation
  \item das Energy Model
  \item multi-path fading modeling (FadingBias)
\end{itemize}
