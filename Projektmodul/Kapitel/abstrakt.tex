
\begin{abstract}
Durch die Erweiterung \textit{Amendment A: MAC sublayer} auf die Version IEEE 802.15.4e
wurden Low-Rate Wireless Personal Area Networks LR-WPANs um mehrere Funktionen erweitert.
Die am meisten diskutierte Erweiterung beschreibt dabei den
\textit{Timeslotted Channel Hopping Mode TSCH},
welcher ein Kanalsprungverfahren implementiert, wodurch
802.15.4-Netzwerke robuster und zuverl"assiger im industriellen Umfeld werden sollen.
Daher besch"aftigt sich diese Arbeit mit dem Entwicklungsrepository, welcher das
bestehende LR-WPAN Modul im Netzwerksimulator ns-3 um die Funktionalit"aten des
TSCH Modes erweitert. Dabei soll der grunds"atzliche Entwicklungsstand analysiert
werden, sowie anhand von Beispielen die m"ogliche Anwendung innerhalb eines
vordefinierten Anwendungsfalles, dem Network Formation Process
im TSCH Mode, erforscht werden. Aufgrund dieser Vorgabe behandelt diese Arbeit
in nachfolgender Reihenfolge die Grundlagen der Theorie und Implementierung des
TSCH Mode, den theoretischen Ablauf des Network Formation Process, wichtige
Kernmethoden und Funktionalit"aten des LR-WPAN TSCH Modul in ns-3, sowie abschliessend
verschiedene experimentelle Anwendungen mit dem Entwicklungsrepository. Damit
dient diese Arbeit als theoretische Vorlage f"ur k"unftige Entwicklungsarbeiten
am Kommunikationssysteme Lehrstuhl der Universit"at Augsburg.

\end{abstract}
