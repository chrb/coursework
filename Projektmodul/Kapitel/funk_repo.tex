\clearpage
\section{Kernfunktionalit"aten in der Entwicklung mit ns-3-dev-TSCH}

Dieses Kapitel beschreibt die Kernfunktionalit"aten welche f"ur den Umgang und
die Entwicklung mit dem Entwicklungsrepository ns-3-dev-TSCH wichtig sind.
Dabei werden zuerst die Kernmethoden aus der ns-3 Helperklasse
\textit{lr-wpan-tsch-helper} erl"autert, bevor weitere allgemeine ns-3
Funktionen n"aher betrachtet werden.

\subsection{Kernmethoden der Helperklasse LrWpanTschHelper}

\subsubsection{LrWpanTschHelper::Install}
Die Methode \textit{LrWpanTschHelper::Install} ab Zeile 309 enh"alt einen
NodeIterator welcher f"ur jeden erstellten Knoten das dazugeh"orige NetDevice
erstellt und ihm den richtigen Channel zuordnet. Wie im Kommentar ersichtlich
fehlt bislang die M"oglichkeit den Adressenmodus sowie die PanID f"ur
die NetDevices bei Erstellung festzulegen, statt dessen muss dieser Vorgang
in einem zus"atzlichen Schritt durchgef"uhrt werden.

\begin{lstlisting}[frame=single]
NetDeviceContainer
LrWpanTschHelper::Install (NodeContainer c)
{
  NetDeviceContainer devices;
  for (NodeContainer::Iterator i = c.Begin (); i != c.End (); i++)
    {
      Ptr<Node> node = *i;

      Ptr<LrWpanTschNetDevice> netDevice = CreateObject<LrWpanTschNetDevice> ();
      netDevice->SetChannel (m_channel);
      node->AddDevice (netDevice);
      netDevice->SetNode (node);
      // \todo add the capability to change short address, extended
      // address and panId. Right now they are hardcoded in LrWpanTschMac::LrWpanTschMac ()
      devices.Add (netDevice);
    }
  return devices;
}
\end{lstlisting}

\subsubsection{LrWpanTschHelper::AssociateToPan}
\label{helper_associatetopan}
Die Methode \textit{LrWpanTschHelper::AssociateToPan} beginnend ab Zeile 390
durchl"auft alle erstellten NetDevices und legt deren PanId und ShortAdresse
f"ur den TSCH (NMAC) und Nicht-TSCH (OMAC) Mode fest. Diese Funktionalit"aten
sollten in jeweils eigene Methoden ausgelagert werden, im besonderen das
Festlegen der Adressen um die hardgecodete Implementierung zu ersetzen.

\begin{lstfloat}
\begin{lstlisting}[frame=single]
void
LrWpanTschHelper::AssociateToPan (NetDeviceContainer c, uint16_t panId)
{
  NetDeviceContainer devices;
  uint16_t id = 1;
  uint8_t idBuf[2];

  for (NetDeviceContainer::Iterator i = c.Begin (); i != c.End (); i++)
    {
      Ptr<LrWpanTschNetDevice> device = DynamicCast<LrWpanTschNetDevice> (*i);
      if (device)
        {
          idBuf[0] = (id >> 8) & 0xff;
          idBuf[1] = (id >> 0) & 0xff;
          Mac16Address address;
          address.CopyFrom (idBuf);

          device->GetOMac ()->SetPanId (panId);
          device->GetOMac ()->SetShortAddress (address);
          device->GetNMac ()->SetPanId (panId);
          device->GetNMac ()->SetShortAddress (address);
          id++;
        }
    }
  return;
}
\end{lstlisting}
\end{lstfloat}

\subsubsection{LrWpanTschHelper::ConfigureSlotframeAllToPan}

Die Methode \textit{LrWpanTschHelper::ConfigureSlotframeAllToPan} ab Zeile 1068
definiert den Schedule der Kommunikation, durch die Reihenfolge der
Link-Methoden \textit{AddLink}, \textit{AddAdvLink} und \textit{AddBcastLinks}.
Die Methode erstellt am Anfang eine Anzahl an Slotframes in Abh"angigkeit
zur Anzahl der NetDevices, Empty Timeslots sowie einem zus"atzlichen Broadcast
und der M"oglichkeit der bidirektionalen Verbindung. Anschliessend werden
ein Advertisment Link im Timeslot 0 hinzugef"ugt, der optionale Broadcast Link
sowie den Direktverbindung an beteiligten NetDevices in einfacher und bidirektionaler
Verbindung.
Diese Methode sowie deren verwendeten Helper-Methoden m"ussen "uberarbeitet
werden, damit ein selbstgew"ahlter Schedule mit definierten Variablen anwendbar ist,
ansonsten k"onnen nur die im Rahmen der Parameter entsprechend Kommunikationsabl"aufe
definiert werden.


\begin{lstlisting}[frame=single]
void
LrWpanTschHelper::ConfigureSlotframeAllToPan(NetDeviceContainer devs, int empty_timeslots, bool bidir, bool bcast)
{
  int size = (bcast ? 2 : 1) + (bidir ? 2 : 1)*(devs.GetN ()-1) + empty_timeslots;

  AddSlotframe(devs,m_slotframehandle,size);

  //Add links
  AddLinkParams alparams;
  alparams.slotframeHandle = m_slotframehandle;
  alparams.channelOffset = 0;

  alparams.linkHandle = 0;
  alparams.timeslot = 0;
  AddAdvLink (devs,0,alparams);

  if (bcast) {
	  alparams.linkHandle = 1;
	  alparams.timeslot = 1;
	  AddBcastLinks (devs,0,alparams);
  }

  uint16_t c=(bcast ? 2 : 1);

  for (u_int32_t i = 0; i < devs.GetN ()-1; i++,c++)
    {
      alparams.linkHandle = c;
      alparams.timeslot = c;
      AddLink(devs,i+1,0,alparams,false);
    }

  if (bidir)
    for (u_int32_t i = 0; i < devs.GetN ()-1; i++,c++)
    {
      alparams.linkHandle = c;
      alparams.timeslot = c;
      AddLink(devs,0,i+1,alparams,false);
    }
   m_slotframehandle++;
}
\end{lstlisting}

\subsubsection{LrWpanTschHelper::AddAdvLink und LrWpanTschHelper::AddBcastLinks}

Die beiden Methoden \textit{LrWpanTschHelper::AddAdvLink} in Zeile 868 und
\textit{LrWpanTschHelper::AddBcastLinks} in Zeile 904 beschreiben den
Advertisment- bzw. Broadcast-Link. Nachfolgend ist der Code f"ur den Advertisment
Link abgebildet. Der Unterschied zwischen beiden Linktypen liegt in der Verarbeitung
der Sende und Empfangsverbindung.
Der Broadcast Link erstellt MlmeSetLinkRequests an alle Knoten und sendet an die
Adresse \textit{ff:ff} w"ahrend der Transmitphase. Im Laufe der Receive Phase
h"ort er auf die Adresse \textit{ff:ff} und "'offnet Links an alle Knoten mit einem
weiteren MlmeSetLinkRequest. Im aktuellen Entwicklungsstand ist der Broadcast Link
nicht anwendbar, da s"aemtliche Links innerhalb des PANs in den Sendenmodus geschaltet werden
und auch diesen w"ahrend des Timeslots nicht verlassen, weshalb zwar Pakete gesendet
aber nicht empfangen werden k"onnen.
F"ur unseren Anwendungsfall soll dieser allerdings den "'Join Request"' eines
beitretenden Knoten beinhalten, weshalb die Implementierung ge"andert werden muss.
Da zumindest der PAN Koordinator die Pakete empfangen muss gehen wir in Zukunft
davon aus, dass der PAN Koordinator das erste NetDevice darstellt. Konkret wird k"unftig
nicht erste NetDevice nicht mehr in den Transmit Modus geschalten, indem die for-Schleife
erst mit dem zweiten NetDevice beginnt zu z"ahlen.

\begin{lstlisting}[frame=single]
//10000000 to transmit
linkRequest.linkOptions.reset();
linkRequest.linkOptions.set(0,1);
linkRequest.linkOptions.set(2,1);
linkRequest.linkType = MlmeSetLinkRequestlinkType_ADVERTISING;
linkRequest.nodeAddr = Mac16Address("ff:ff");
for ( u_int32_t i = 1;i < devs.GetN ();i++)
  {
        linkRequest.TxID = i;
        linkRequest.RxID = coordinatorPos;
        linkRequest.linkFadingBias = FadingBias[coordinatorPos][i];
        devs.Get(i)->GetObject<LrWpanTschNetDevice> ()->GetNMac()->MlmeSetLinkRequest (linkRequest);
  }
\end{lstlisting}

Der Advertisment Link dagegen sendet in der Transmit Phase ebenfalls an die
Adresse \textit{ff:ff}, erstellt aber nur einen MlmeSetLinkRequest an ein bestimmten
NetDevice. Im Verlauf der Receive Phase wird nur auf die Adresse des Senders
geh"ort und nur zu diesem ein Link aufgebaut.

Die Methode \textit{LrWpanTschHelper::AddAdvLink} wird f"ur den Network Formation
Process ben"otigt und erh"aelt in der Variable \textit{u\_int32\_t senderPos}
das beizutretende NetDevice.

Aufgrund der momentanen Implementierung, k"onnen wir den Broadcast Link leider
nicht f"ur Broadcast Transmits verwenden, weshalb der Advertisment Link
diese Aufgabe "'ubernehmen wird, solange dieser Umstand nicht ge"andert wird.


\begin{lstlisting}[frame=single]
void
LrWpanTschHelper::AddAdvLink(NetDeviceContainer devs,u_int32_t senderPos, AddLinkParams params)
{
  MlmeSetLinkRequestParams linkRequest;
  linkRequest.Operation = MlmeSetLinkRequestOperation_ADD_LINK;
  linkRequest.linkHandle = params.linkHandle;
  linkRequest.slotframeHandle = params.slotframeHandle;
  linkRequest.Timeslot = params.timeslot;
  linkRequest.ChannelOffset = params.channelOffset;

  //10000000 to transmit
  linkRequest.linkOptions.reset();
  linkRequest.linkOptions.set(0,1);
  linkRequest.linkType = MlmeSetLinkRequestlinkType_ADVERTISING;
  linkRequest.nodeAddr = Mac16Address("ff:ff");
  linkRequest.linkFadingBias = NULL;
  linkRequest.TxID = senderPos;
  linkRequest.RxID = 0;
  devs.Get(senderPos)->GetObject<LrWpanTschNetDevice> ()->GetNMac()->MlmeSetLinkRequest (linkRequest);

  //01010000 to receive
  linkRequest.linkOptions.reset();
  linkRequest.linkOptions.set(1,1);
  linkRequest.linkOptions.set(3,1);
  linkRequest.nodeAddr = Mac16Address::ConvertFrom(devs.Get(senderPos)->GetAddress());
  for ( u_int32_t i = 0;i < devs.GetN ();i++)
      if (i != senderPos) {
          linkRequest.linkFadingBias = FadingBias[i][senderPos];
          linkRequest.TxID = senderPos;
          linkRequest.RxID = i;
          devs.Get(i)->GetObject<LrWpanTschNetDevice> ()->GetNMac()->MlmeSetLinkRequest (linkRequest);
        }
}
\end{lstlisting}

\subsubsection{LrWpanTschHelper::GenerateTraffic}

Mit den bislang beschriebenen Hilfsmethoden wird der Schedule innerhalb des TSCH
Mode fest und entsprechen damit den M"oglichkeiten die h"ohere Schichten besitzen
um Routing und Datenverkehr zu beschreiben. Dadurch wird zwar vorgegeben wann und
wie die Kommunikation innerhalb eines PAN stattfindet, der eigentliche Austausch von
Paketen zwischen den Knoten wird aber von einer anderen Hilfsmethode geregelt.

Diese Funktionalit"at wird von der Methode \textit{LrWpanTschHelper::GenerateTraffic}
ab Zeile 1129 wiedergespiegelt. Dabei wird ein Simulator Schedule gestartet, welcher
einen Ablauf an zeitdiskreten Events beschreibt wodurch MAC-Frames zwischen dem
ausgew"ahlten NetDevice \textit{dev} und seinem Ziel \textit{dst} gesendet werden.
Die Parameter \textit{start} und \textit{duration} geben dabei die Gesamtl"ange
der Kommunikation an. Wichtig ist zus"atzlich der Parameter \textit{interval}, welcher
die zeitlichen Abst"ande zwischen zwei Events angibt. Die Gewichtung dieses Parameter
wird im

\begin{lstlisting}[frame=single]
void
LrWpanTschHelper::GenerateTraffic(Ptr<NetDevice> dev, Address dst, int packet_size, double start, double duration, double interval)
{
  double end = start+duration;
  Simulator::Schedule(Seconds(start),&LrWpanTschHelper::SendPacket,this,dev,dst,packet_size,interval,end);
}
\end{lstlisting}


\subsection{Logging und Tracing}

Das Logging in ns3 erm"oglicht genaue Informationen "'uber den genauen Ablauf
der Netzwerksimulation zu erhalten wird das Logging eingesetzt, damit k"onnen
genauere Informationen erhalten werden, als beim "'ublichen Tracing. Da die
Informationsf"ulle extrem ansteigt wurde f"ur die einzelnen Loggingkomponenten
eine Hilfsmethode geschrieben.

\begin{lstlisting}[frame=single]
void
LogComponents(bool phy, bool csmaca, bool diverse)
{
  LogComponentEnableAll (LOG_PREFIX_TIME);
  LogComponentEnableAll (LOG_PREFIX_FUNC);
  LogComponentEnable ("LrWpanTschMac", LOG_LEVEL_ALL);
  LogComponentEnable ("LrWpanTschNetDevice", LOG_LEVEL_ALL);
  if (phy)
  {
  LogComponentEnable ("LrWpanPhy", LOG_LEVEL_ALL);
  LogComponentEnable ("LrWpanSpectrumSignalParameters", LOG_LEVEL_ALL);
  LogComponentEnable ("LrWpanSpectrumValueHelper", LOG_LEVEL_ALL);
  }
  if (csmaca)
  {
  LogComponentEnable ("LrWpanCsmaCa", LOG_LEVEL_ALL);
  }
  if (diverse)
  {
  LogComponentEnable ("LrWpanErrorModel", LOG_LEVEL_ALL);
  LogComponentEnable ("LrWpanInterferenceHelper", LOG_LEVEL_ALL);
  }
}
\end{lstlisting}

Als Beispiel wird ein Ausschnitt herangezogen, der am Anfang eines neuen diskreten
Events steht und die TSCH Funktionalit"at der Simulation angibt.
Darin meldet zuerst das LrWpanTschNetDevice \textit{00:02}, dass Daten an das Ger"at
mit der MAC-Adresse \textit{02-02-00:01} gesendet werden. Nachdem die Primitiven
gesetzt wurden, inkrementieren zuerst alle Ger"ate iher ASN und geben ihre
Aktion in diesem Timeslot an. Im Beispiel empf"angt \textit{00:01} Daten und
richtet sich darauf ein, diese zu Empfangen. Das Ger"at \textit{00:02} erkennt ein
Paket in seiner Warteliste und startet die Sendeprozedur.

\begin{lstlisting}[frame=single]
0.02s LrWpanTschNetDevice:Send(0x14ee770, 0x14fd5f0, 02-02-00:01, 34525)
0.02s LrWpanTschNetDevice:GetMtu(0x14ee770)
0.02s [address 00:02] LrWpanTschMac:McpsDataRequest(0x14ee800, 0x14fd5f0)
0.02s [address 00:02] LrWpanTschMac:SetTxLinkQueue(0x14ee800)
0.02s [address 00:02] LrWpanTschMac:SetTxLinkQueue(): Enqueuing packet with SeqNum = 108 in queue with link sequence = 0
0.02s [address 00:01] LrWpanTschMac:IncAsn(0x14ec530)
0.02s [address 00:02] LrWpanTschMac:IncAsn(0x14ee800)
0.02s [address 00:03] LrWpanTschMac:IncAsn(0x14f0c50)
0.02s [address 00:04] LrWpanTschMac:IncAsn(0x14f3080)
0.02s [address 00:05] LrWpanTschMac:IncAsn(0x14f54e0)
0.02s [address 00:06] LrWpanTschMac:IncAsn(0x14f7980)
0.02s [address 00:07] LrWpanTschMac:IncAsn(0x14f9da0)
0.02s [address 00:08] LrWpanTschMac:IncAsn(0x14fc1f0)
0.02s [address 00:01] LrWpanTschMac:ScheduleTimeslot(): Timeslot 2 ts = 2 Queue size = 0
0.02s [address 00:01] LrWpanTschMac:ScheduleTimeslot(): Link found at timeslot 2
0.02s [address 00:01] LrWpanTschMac:ScheduleTimeslot(): TSCH Changing to channel 18
0.02s [address 00:01] LrWpanTschMac:ScheduleTimeslot(): 0x14ec530setting for channel 18 fading bias: 1
0.02s [address 00:01] LrWpanTschMac:ScheduleTimeslot(): Start timeslot receiving procedure
0.02s [address 00:02] LrWpanTschMac:ScheduleTimeslot(): Timeslot 2 ts = 2 Queue size = 1
0.02s [address 00:02] LrWpanTschMac:ScheduleTimeslot(): Link found at timeslot 2
0.02s [address 00:02] LrWpanTschMac:ScheduleTimeslot(): TSCH Changing to channel 18
0.02s [address 00:02] LrWpanTschMac:ScheduleTimeslot(): 0x14ee800setting for channel 18 fading bias: 1
0.02s [address 00:02] LrWpanTschMac:ScheduleTimeslot(): Queue contained link size = 1
0.02s [address 00:02] LrWpanTschMac:FindTxPacketInEmptySlot(0x14ee800)
0.02s [address 00:02] LrWpanTschMac:FindTxPacketInEmptySlot(): Find Tx packet in queue with link position = 0 with queue size = 1
0.02s [address 00:02] LrWpanTschMac:ScheduleTimeslot(): Start timeslot transmiting procedure, seqnum = 108
0.02s [address 00:03] LrWpanTschMac:ScheduleTimeslot(): Timeslot 2 ts = 2 Queue size = 0
0.02s [address 00:03] LrWpanTschMac:ScheduleTimeslot(): No link in this timeslot, turning off the radio
0.02s [address 00:04] LrWpanTschMac:ScheduleTimeslot(): Timeslot 2 ts = 2 Queue size = 0
0.02s [address 00:04] LrWpanTschMac:ScheduleTimeslot(): No link in this timeslot, turning off the radio
0.02s [address 00:05] LrWpanTschMac:ScheduleTimeslot(): Timeslot 2 ts = 2 Queue size = 0
0.02s [address 00:05] LrWpanTschMac:ScheduleTimeslot(): No link in this timeslot, turning off the radio
0.02s [address 00:06] LrWpanTschMac:ScheduleTimeslot(): Timeslot 2 ts = 2 Queue size = 0
0.02s [address 00:06] LrWpanTschMac:ScheduleTimeslot(): No link in this timeslot, turning off the radio
0.02s [address 00:07] LrWpanTschMac:ScheduleTimeslot(): Timeslot 2 ts = 2 Queue size = 0
0.02s [address 00:07] LrWpanTschMac:ScheduleTimeslot(): No link in this timeslot, turning off the radio
0.02s [address 00:08] LrWpanTschMac:ScheduleTimeslot(): Timeslot 2 ts = 2 Queue size = 0
0.02s [address 00:08] LrWpanTschMac:ScheduleTimeslot(): No link in this timeslot, turning off the radio
\end{lstlisting}


Zum tracen der einzelnen Pakete, wird allen vorran die Methode \textit{EnablePcap}
verwendet um die pcap Dateien mitzuschneiden, welche anschliessend zur Netzwerkanalyse
in Wireshark ausgewertet werden k"onnen.
\begin{lstlisting}[frame=single]
lrWpanHelper.EnablePcap (std::string ("tsch_scenario"), panCoordNetDevice, true);
\end{lstlisting}
