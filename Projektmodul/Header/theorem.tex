% Mathematische Zeichens\"atze und Umgebungen
\usepackage{amsfonts, amssymb}	% Definition einer Liste mathematischer Fontbefehle und Symbole
\usepackage[intlimits,		% Integralgrenzen \"uber und unter dem Integral
	    sumlimits]		% Summationsgrenzen \"uber und unter der Summe
           {amsmath}		% mathematische Verbesserungen
\usepackage{amsthm}		% spezielle theorem Stile
\usepackage{aliascnt}
%\usepackage{ntheorem-hyper}

% Code
\usepackage{listings}
\usepackage{color}

\definecolor{mygreen}{rgb}{0,0.6,0}
\definecolor{mygray}{rgb}{0.5,0.5,0.5}
\definecolor{mymauve}{rgb}{0.58,0,0.82}

\usepackage{float}
\usepackage{listings}

\newfloat{lstfloat}{htbp}{lop}
\floatname{lstfloat}{Listing}

\lstset{ %
  backgroundcolor=\color{white},   % choose the background color; you must add \usepackage{color} or \usepackage{xcolor}
  basicstyle=\footnotesize,        % the size of the fonts that are used for the code
  breakatwhitespace=false,         % sets if automatic breaks should only happen at whitespace
  breaklines=true,                 % sets automatic line breaking
  captionpos=b,                    % sets the caption-position to bottom
  commentstyle=\color{mygreen},    % comment style
  deletekeywords={...},            % if you want to delete keywords from the given language
  escapeinside={\%*}{*)},          % if you want to add LaTeX within your code
  extendedchars=true,              % lets you use non-ASCII characters; for 8-bits encodings only, does not work with UTF-8
  frame=single,	                   % adds a frame around the code
  keepspaces=true,                 % keeps spaces in text, useful for keeping indentation of code (possibly needs columns=flexible)
  keywordstyle=\color{blue},       % keyword style
  language=c++,                 % the language of the code
  otherkeywords={*,...},           % if you want to add more keywords to the set
  numbers=left,                    % where to put the line-numbers; possible values are (none, left, right)
  numbersep=5pt,                   % how far the line-numbers are from the code
  numberstyle=\tiny\color{mygray}, % the style that is used for the line-numbers
  rulecolor=\color{black},         % if not set, the frame-color may be changed on line-breaks within not-black text (e.g. comments (green here))
  showspaces=false,                % show spaces everywhere adding particular underscores; it overrides 'showstringspaces'
  showstringspaces=false,          % underline spaces within strings only
  showtabs=false,                  % show tabs within strings adding particular underscores
  stepnumber=2,                    % the step between two line-numbers. If it's 1, each line will be numbered
  stringstyle=\color{mymauve},     % string literal style
  tabsize=2,	                   % sets default tabsize to 2 spaces
  title=\lstname                   % show the filename of files included with \lstinputlisting; also try caption instead of title
}

%-----------------------------------------------------------------------------------
% Hilfreiche Befehle
%-----------------------------------------------------------------------------------
\newcommand{\betrag}[1]{\lvert #1 \rvert}	        % Betrag
\newcommand{\norm}[1]{\lVert #1 \rVert}		        % Norm
\providecommand*{\Lfloor}{\left\lfloor}                 % gro\ss{}es Abrunden
\providecommand*{\Rfloor}{\right\rfloor}                % gro\ss{}es Abrunden
\providecommand*{\Floor}[1]{\Lfloor #1 \Rfloor}         % gro\ss{}es ganzes Abrunden
\providecommand*{\Ceil}[1]{\left\lceil #1 \right\rceil} % gro\ss{}es ganzes Aufrunden
\newcommand{\eins}{ \mathds{1} }                        % Einsvektor

\newcommand{\R}{\mathbb{R}}

\DeclareMathOperator{\e}{ex}
\DeclareMathOperator{\ma}{mate}
\DeclareMathOperator{\Ex}{Ex}

%-----------------------------------------------------------------------------------
%   Befehle f\"ur Nummerierung der Ergebnisse
%   fortlaufend innerhalb eines Abschnittes
%-----------------------------------------------------------------------------------
\theoremstyle{plain}            % normaler Stil
\newtheorem{theorem}{Theorem}[section]
% Lemma
\newaliascnt{lemma}{theorem}
\newtheorem{lemma}[lemma]{Lemma}
\aliascntresetthe{lemma}
% Satz
\newaliascnt{satz}{theorem}
\newtheorem{satz}[satz]{Satz}
\aliascntresetthe{satz}
% Korollar
\newaliascnt{korollar}{theorem}
\newtheorem{korollar}[korollar]{Korollar}
\aliascntresetthe{korollar}
% Proposition
\newaliascnt{proposition}{theorem}
\newtheorem{proposition}[proposition]{Proposition}
\aliascntresetthe{proposition}
%-----------------------------------------------------------------------------------
\theoremstyle{definition}	% Definitionsstil
% Definition
\newaliascnt{definition}{theorem}
\newtheorem{definition}[definition]{Definition}
\aliascntresetthe{definition}
% Beispiel
\newaliascnt{beispiel} {theorem}
\newtheorem{beispiel}[beispiel]{Beispiel}
\aliascntresetthe{beispiel}
% Problem
\newaliascnt{problem}{theorem}
\newtheorem{problem}[problem]{Problem}
\aliascntresetthe{problem}
% Algorithmus
\newaliascnt{algorithmus}{theorem}
\newtheorem{algorithmus}[algorithmus]{Algorithmus}
\aliascntresetthe{algorithmus}
%-----------------------------------------------------------------------------------
\theoremstyle{remark}		% Bemerkungsstil
% Bemerkung
\newaliascnt{bemerkung}{theorem}
\newtheorem{bemerkung}[bemerkung]{Bemerkung}
\aliascntresetthe{bemerkung}
% Vermutung
\newaliascnt{vermutung}{theorem}
\newtheorem{vermutung}[vermutung]{Vermutung}
\aliascntresetthe{vermutung}
% Notation
\newaliascnt{notation}{theorem}
\newtheorem{notation}[notation]{Notation}
\aliascntresetthe{notation}

%-----------------------------------------------------------------------------------
% automatische Referenzen mit interaktiven Text
%-----------------------------------------------------------------------------------

% Texte
\newcommand{\theoremautorefname}{Theorem}
\newcommand{\lemmaautorefname}{Lemma}
\newcommand{\satzautorefname}{Satz}
\newcommand{\korollarautorefname}{Korollar}
\newcommand{\propositionautorefname}{Proposition}

\newcommand{\definitionautorefname}{Definition}
\newcommand{\beispielautorefname}{Beispiel}
\newcommand{\problemautorefname}{Problem}
\newcommand{\algorithmusautorefname}{Algorithmus}

\newcommand{\bemerkungautorefname}{Bemerkung}
\newcommand{\vermutungautorefname}{Vermutung}
\newcommand{\notationautorefname}{Notation}

%-----------------------------------------------------------------------------------
% Nummerierung der Gleichungen innerhalb der obersten Ebene
%-----------------------------------------------------------------------------------
\ifx\chapter\undefined 			% Kapitel sind definiert
  \numberwithin{equation}{section}	% Gleichungsnummern in Section
\else					% Kapitel sind nicht definiert
  \numberwithin{equation}{chapter}	% Gleichungsnummern in Kapiteln
\fi
