%-----------------------------------------------------------------------------------
% zweiseitige Kopf-Zeilen
%-----------------------------------------------------------------------------------

\usepackage[automark]{scrpage2}	% Seiten-Stil f\"ur scrartcl
\pagestyle{scrheadings}		% Kopfzeilen nach scr-Standard
\ifx\chapter\undefined 		% falls Kapitel nicht definiert sind
  \automark[subsection]{section}% Kopf- und Fusszeilen setzen
\else				% Kapitel sind definiert
  \automark[section]{chapter}	% Kopf- und Fusszeilen setzen
\fi

%-----------------------------------------------------------------------------------
%   Maske f\"ur \"Uberschrift
%-----------------------------------------------------------------------------------
% Belegung der notwendigen Kommandos f\"ur die Titelseite
\newcommand{\autor}{Christoph Biesinger} 		% bearbeitender Student
\newcommand{\veranstaltung}{Projektmodul} 	% Titel des ganzen Seminars
\newcommand{\uni}{Institut f\"ur Informatik der Universit\"at Augsburg} % Universit\"at
\newcommand{\lehrstuhl}{Kommunikationssysteme} % Lehrstuhl
\newcommand{\semester}{}	% Winter- oder Sommersemester mit Jahr
\newcommand{\datum}{} 			% Datumsabgabe
\newcommand{\thema}{Analyse und Anwendung des Entwicklungsrepository ns-3-dev-TSCH}  		% Titel der Seminararbeit

\newcommand{\ownline}{\vspace{.7em}\hrule\vspace{.7em}} % horizontale Linie mit Abstand

\newcommand{\seminarkopf}{	% Befehl zum Erzeugen der Titelseite f\"ur kleinere Arbeiten
				% wie Seminarausarbeiten und numerisches Praktikum
 \textsc{\autor}  \hfill{\datum} \\
\textbf{\veranstaltung} \\
\uni \\
\lehrstuhl \\
\semester
\ownline

\begin{center}
{\LARGE \textbf{\thema}}
\end{center}
\ownline\vspace{2em} \vspace{2em}
\noindent
\ownline

}
%-----------------------------------------------------------------------------------

%-----------------------------------------------------------------------------------
%   Maske f\"ur neuen Abschnitt
%   nur bei f\"ur besondere Nummerierung notwendig,
%   zum Beispiel wenn die Kapitelnummer nicht bei 1 starten soll
%-----------------------------------------------------------------------------------
\newcommand {\abschnitt} [2]		% \"Uberschriftenbefehl f\"ur eigene Nummerierungen
{\setcounter{section}{#1}
 \setcounter{theorem}{0}
 \setcounter{equation}{0}
 \vspace{4ex}
 {\large \textbf{#1. #2}}
\vspace{1ex}}


% -----
% Bilder
% -----
%\graphicspath{{./images/}}
