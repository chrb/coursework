\section{Networksimulator ns-3} \label{Kap5-5}

Im Laufe der Projekts wird der Netzwerksimulator ns-3 verwendet, dabei
kann die Grobbeschreibung direkt von dem Internetauftritt unter
\href{http://www.nsam.org}{http://www.nsam.org} entnommen werden.

\begin{alltt}
ns-3 is a discrete-event network simulator for Internet systems,
targeted primarily for research and educational use.
ns-3 is free software, licensed under the GNU GPLv2 license,
and is publicly available for research, development, and use.
\end{alltt}

ns-3 ist konzipiert als eine Sammlung von Software-Bibliotheken, welche zusammen
ein System bilden. Diese einzelnen Bibliotheken werden als Module bezeichnet
und implementieren jeweils einen eigenen Standard, welche auf den Kernmodulen
aufbauen. Dadurch k"onnen Benutzerprogramme durch Verwendung dieser Module
aufgebaut werden.

Wir gehen im weiteren davon aus, dass der Leser mit ns-3 vertraut ist,
falls nicht, kann das mit Hilfe des \href{https://www.nsnam.org/ns-3-24/documentation/}{Tutorial}
nachgeholt werden.

%----------------------------------------------
\subsection{LR-WPAN Modul}

Im Verlauf dieser Arbeit wurde haupts"achlich das LR-WPAN Modul angewendet, welches
in der offiziellen ns-3 Version (NS-3.24) den IEEE 802.15.4.e Standard mitsamt
dessen Funktionalit"aten und Hilfsanwendungen implementiert.

Der Auf

F"ur diese Arbeit ist dabei das Lr-Wpan Modul wichtig, welches im aktueller Version
(NS-3.24) den 802.15.4 Standard mitsamt seiner Funktionalit"aten implementiert.
Der Aufbau des Modul entspricht dabei sehr stark dem IEEE Standard.

Der Aufbau der Implementierung erfolgt dabei nach NS3 Standard und entspricht
den Ordnern \textit{models}, \textit{examples}, \textit{helper} und \textit{test}.
Im besonderen ist der Ordner \textit{models} zu betrachten, da dieser die Implementierung
des Standards enth"alt.

F"ur eine genauere Beschreibung und allgemeine Informationen zur Arbeit und
Entwicklung im Rahmen des LR-WPAN Moduls kann auf die Arbeit von
Nikica Krezic-Luger\cite{bachelorarbeit} verwiesen werden.

%----------------------------------------------
\subsection{TSCH Erweiterung}

Durch das Amendment A auf Version 802.15.4e muss auch ns-3 um die neuen
Funktionen erweitert werden. Daf"ur wurde mit einem Projekt begonnen, welches
im Januar 2015 zu einem ersten gro"sen Commit f"uhrte. Dieser wurde anschlie"send
f"ur eine Codereview freigestellt und kann in Github unter
\href{https://github.com/EIT-ICT-RICH/ns-3-dev-TSCH}{ns-3-dev-TSCH}
gefunden werden.

Dieser Commitstand umfasst dabei laut der
\href{http://mailman.isi.edu/pipermail/ns-developers/2015-January/012459.html}{Ver"offentlichsmeldung}
folgende Funktionen:

\begin{itemize}
  \item Wechsel von Standard 802.15.4 auf den TSCH Mode, w"ahrend der Kommunikation
  \item das Energy Model
  \item multi-path fading modeling (FadingBias)
\end{itemize}

Diese und weitere Funktionen f"uhrten zu "Anderungen innerhalb des LR-WPAN Models,
haupts"achlich innerhalb des lr-wpan-mac Models sowie durch die Hinzunahme
der weiteren Models

\begin{itemize}
  \item lr-wpan-array
  \item lr-wpan-tsch-mac
  \item lr-wpan-tsch-net-device
\end{itemize}

%----------------------------------------------
