\subsection{IEEE 802.15.4e Amendment A: MAC sublayer} \label{Kap5-3}

Um die Probleme von 802.15.4 zu beheben wurde die 802.15.4e Working Group geschaffen,
welche das existierende MAC Protokoll verbessern sollte. Dadurch wurde
der 802.15.4 Standard um zwei Kategorien an MAC Verbesserungen erweitert.
Einmal den MAC Behavior Modes, welche spezifische Anwendungsbereiche verbessern,
sowie General Functional Enhancements f"ur generelle Verbesserungen. Bei der Entwicklung
wurden dabei viele Ideen aus bereits existierenden Standards wie WirelessHART und
ISA 100.11.a "ubernommen.

\begin{description}

\item[General Functional Enhancements] \hfill \\
  \begin{description}
    \item[Information Elements IE] sind bereits seit der Grundversion implementiert,
    "ubernehmen aber zus"atzliche Aufgaben im Amendment A.
    \item[Enhanced Beacons EB] bilden eine Erweiterung zu Beacon Frames mit gr"osserer
    Flexibili"at. Werden als anwendungsbezogene Beacons via IEs im TSCH Mode verwendet.

    \item[Low Energy LE] Austausch der Eigenschaften niedrige Latenz durch
    niedriger Energieverbrauch, wodurch das Ger"at immer erreichbar erscheint
%      \item[Multipurpose Frame] addressiert mehrere MAC Operations
%      \item[MAC Performance Metrics] stellen einen Mechanismus bereit um die
%      Kanalqualit"at zu ermitteln, wird f"ur das Networking sowie h"ohere Schichten
%      (6LoWPAN, RPL) ben"otigt.
%      \item[Fast Assocation FastA] wird bei zeit-kritischen Systemen angewendet.
  \end{description}

  \item[Behavior Modes] \hfill \\

    \begin{description}
      \item[Timeslotted Channel Hopping TSCH] f"ur die Verbesserung von automatisierten
      Anwendungsbereichen.
      \item[Deterministic and Synchronous Multi-Channel Extension DSME] zur Unterst"utzung
      von industriellen und kommerziellen Anwendungen, welche strikte Vorgaben
      an Zeit und Verf"ugbarkeit stellen.
    \end{description}
\end{description}
