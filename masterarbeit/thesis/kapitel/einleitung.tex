\section{Einleitung}
\label{sec:einleitung}

Lorem ipsum dolor sit amet, \cite{gof,mde:uml211:07b} consectetuer adipiscing elit,
sed diam nonummy nibh euismod tincidunt ut laoreet dolore magna aliquam erat volutpat.
 Ut wisi enim ad minim veniam, quis nostrud exerci tation ullamcorper suscipit
 lobortis nisl ut aliquip ex ea commodo consequat.


% ------------------------------------------------
% Problemstellung
% ------------------------------------------------
\subsection{Problemstellung}
\label{subsec:einleitung_problemstellung}

Eine der Hauptaufgaben des Data Link Layers ist der Aufbau einer Knoten-zu-Knoten
Datenverbindung anhand der physikalischen Infrastruktur auch Link genannt.
Der logische Topologieaufbau soll daher erst auf dem Network Layer mithilfe
eines Routingprotokolls bewerkstelligt werden. IEEE802.15.4 dagegen zwingt bereits
auf dem Data Link Layer dem Netzwerk eine Topologie auf. Diese Topologien, namentlich
Stern, Peer-to-Peer und Cluster-Tree, belegen aufgrund ihrer unterschiedlichen
Eigenschaften das Routingprotokoll RPL mit Beschr"ankungen bei dem Topologieaufbau
und damit insgesamt schlechteren Netzwerkleistungen.

Daher wird in dieser Arbeit ein L"osungsansatz mithilfe eines TSCH Scheduling Algorithmus
zur Behebung dieses Problems vorgestellt.

% ------------------------------------------------
% Fragestellung und Ziele
% ------------------------------------------------
\subsection{Ziele}
\label{subsec:einleitung_ziele}

% ------------------------------------------------
% Grober Loesungsansatz
% ------------------------------------------------
\subsection{L"osungsansatz}
\label{subsec:einleitung_loesungsansatz}
