\section{Analyse und Evaluation}
\label{sec:evaluation}

% ------------------------------------------------
% Prozess
% ------------------------------------------------

\subsection{Testprozess in Cooja/Contiki}
\label{subsec:ae_testprozess}

Der Testablauf wird zwei Bereiche aufgeteilt, den

\begin{itemize}
  \item Netzwerkaufbau
  \item Beitrittsprozess
\end{itemize}

\subsection{Topologien}
\label{subsec:ae_topologien}

F"ur Testzwecke wurden zwei Netzwerktopologien ausgew"ahlt um den Algorithmus mit
mehreren Anwendungsf"allen zu kontrollieren

Netzwerk A
Platzhalter - Durch Bild ersetzen
\begin{lstlisting}[frame=single]
//
// n0 - n1 - n2
//  |    |    |
// n3 - n4 - n5
//  |    |    |
// n6 - n7 - n8
//
\end{lstlisting}

Netzwerk B
Knoten n6 betritt das bereits etablierte Netzwerk mit zwei Parentknoten
Platzhalter - Durch Bild ersetzen
\begin{lstlisting}[frame=single]
//
//          n0
//         /  \
//      n1     n2
//   / | | \  /
// n3 n4 n5 n6
//
\end{lstlisting}

\subsection{Leistungsmetriken}
\label{subsec:ae_leistungsmetriken}


\begin{description}
  \item [Network Formation Time] Zeit vom ersten Enhanced Beacon, bis letzter
  Knoten dem Netzwerk beigetreten ist
  \item [Joining Time] Zeit die ein Knoten ben"otigt um einem Netzwerk beizutreten
  (abh"angig von festgelegten Regeln)
  \item [Overhead] Welchen Overhead an Nachrichten erzeugt der Algorithmus
  \item [Zeitraum Kontrollnachrichten] Wieviel Zeit wird f"ur die Kontrollnachrichten
  reserviert
  \item [Verhalten Algorithmus "'uber l"angeren Zeitraum] Wie verhalten sich die Leistungsmetriken
  "'uber einen l"angeren Zeitraum?
  \item [Aktivit"atszeit Knoten] Wie lange ist ein Knoten im Rahmen des Schedules aktiv
  im Verh"altnis zur tats"achlichen Aktivit"at.
  \item [Energieverbrauch] Wieviel Energie
\end{description}

% ------------------------------------------------
% Referenzwerte Minimal 6TiSCH
% ------------------------------------------------
\subsection{Network Formation mit Minimal 6TiSCH Configuration}
\label{subsec:ae_6tisch}

% ------------------------------------------------
% Ergebnisse Algorithmus
% ------------------------------------------------
\subsection{Network Formation mit Algorithmus}
\label{subsec:ae_algorithmus}

% ------------------------------------------------
% Vergleich und Evaluation
% ------------------------------------------------
\subsection{Bewertung der Ergebnisse}
\label{subsec:ae_bewertung}
