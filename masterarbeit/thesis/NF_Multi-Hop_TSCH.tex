\documentclass[final]{lktseminar}

% ------------------------------------------------
%  Preamble / Praeambel
% ---------------------
%Text-Encoding
%\usepackage[latin1]{inputenc} %latin1
%\usepackage[utf8]{inputenc} %UTF-8

%Stil der Zitate und der Bibliographie
\usepackage[style=numeric-comp]{biblatex}
%Bibliographie laden:
\addbibresource{literatur/literature.bib}

%
\graphicspath{{./images/}}

% Satz aufhuebschen
\clubpenalty=10000  % Keine Schusterjungen, Hurens"ohne
\widowpenalty=10000

%Silbentrennung:
% Liste von unbekannten und zusammengesetzten Woertern.
%Bei Woerter mit Bindestrich ist das ...-... durch ...''=... zu ersetzen -> weitere Informationen:
% http://de.wikibooks.org/wiki/LaTeXWoerterbuch:_Silbentrennung
\hyphenation {%
}
% ------------------------------------------------
% Code
\usepackage{listings}
\usepackage{color}

\definecolor{mygreen}{rgb}{0,0.6,0}
\definecolor{mygray}{rgb}{0.5,0.5,0.5}
\definecolor{mymauve}{rgb}{0.58,0,0.82}

\usepackage{float}
\usepackage{listings}

\newfloat{lstfloat}{htbp}{lop}
\floatname{lstfloat}{Listing}

\lstset{ %
  backgroundcolor=\color{white},   % choose the background color; you must add \usepackage{color} or \usepackage{xcolor}
  basicstyle=\footnotesize,        % the size of the fonts that are used for the code
  breakatwhitespace=false,         % sets if automatic breaks should only happen at whitespace
  breaklines=true,                 % sets automatic line breaking
  captionpos=b,                    % sets the caption-position to bottom
  commentstyle=\color{mygreen},    % comment style
  deletekeywords={...},            % if you want to delete keywords from the given language
  escapeinside={\%*}{*)},          % if you want to add LaTeX within your code
  extendedchars=true,              % lets you use non-ASCII characters; for 8-bits encodings only, does not work with UTF-8
  frame=single,	                   % adds a frame around the code
  keepspaces=true,                 % keeps spaces in text, useful for keeping indentation of code (possibly needs columns=flexible)
  keywordstyle=\color{blue},       % keyword style
  language=c++,                 % the language of the code
  otherkeywords={*,...},           % if you want to add more keywords to the set
  numbers=left,                    % where to put the line-numbers; possible values are (none, left, right)
  numbersep=5pt,                   % how far the line-numbers are from the code
  numberstyle=\tiny\color{mygray}, % the style that is used for the line-numbers
  rulecolor=\color{black},         % if not set, the frame-color may be changed on line-breaks within not-black text (e.g. comments (green here))
  showspaces=false,                % show spaces everywhere adding particular underscores; it overrides 'showstringspaces'
  showstringspaces=false,          % underline spaces within strings only
  showtabs=false,                  % show tabs within strings adding particular underscores
  stepnumber=2,                    % the step between two line-numbers. If it's 1, each line will be numbered
  stringstyle=\color{mymauve},     % string literal style
  tabsize=2,	                   % sets default tabsize to 2 spaces
  title=\lstname                   % show the filename of files included with \lstinputlisting; also try caption instead of title
}


% ------------------------------------------------
%  ... / Dokument

\begin{document}
% ------------------------------------------------
% .../Informationen
% -----------------
\myname{Biesinger Christoph}
\mytitle{Network Formation in Multi-Hop TSCH Netzwerken}
\myshorttitle{Network Formation in Multi-Hop TSCH Netzwerken}
\mymatnr{1350722}
\mystudycourse{Masterstudiengang Informatik}
\myreviewer{Dario Fanucchi}
% ------------------------------------------------

% Setting of language
\selectlanguage{ngerman}
%\selectlanguage{english}

\makeAuthor
\makeTitle
\makeInfos
\maketitle

% ------------------------------------------------
% Abstract / Kurzfassung
% ----------------------
\begin{abstract}
\section*{Kurzfassung}
Lorem ipsum dolor sit amet, consectetuer adipiscing elit, sed diam nonummy nibh euismod tincidunt ut laoreet dolore magna aliquam erat volutpat. Ut wisi enim ad minim veniam, quis nostrud exerci tation ullamcorper suscipit lobortis nisl ut aliquip ex ea commodo consequat. Duis autem vel eum iriure dolor in hendrerit in vulputate velit esse molestie consequat, vel illum dolore eu feugiat nulla facilisis at vero et accumsan et iusto odio dignissim qui blandit praesent luptatum zzril delenit augue duis dolore te feugait nulla facilisi. Nam liber tempor cum soluta nobis eleifend option congue nihil imperdiet doming id quod mazim placerat facer possim assum.
\end{abstract}

\newpage
% ------------------------------------------------


% ------------------------------------------------
% Inhaltverzeichnis
% ------------------------------------------------
\tableofcontents
\newpage
% ------------------------------------------------


% ------------------------------------------------
% Einleitung
% ------------------------------------------------
\section{Einleitung}
\label{sec:einleitung}

Lorem ipsum dolor sit amet, \cite{gof,mde:uml211:07b} consectetuer adipiscing elit,
sed diam nonummy nibh euismod tincidunt ut laoreet dolore magna aliquam erat volutpat.
 Ut wisi enim ad minim veniam, quis nostrud exerci tation ullamcorper suscipit
 lobortis nisl ut aliquip ex ea commodo consequat.


% ------------------------------------------------
% Problemstellung
% ------------------------------------------------
\subsection{Problemstellung}
\label{subsec:einleitung_problemstellung}

Eine der Hauptaufgaben des Data Link Layers ist der Aufbau einer Knoten-zu-Knoten
Datenverbindung anhand der physikalischen Infrastruktur auch Link genannt.
Der logische Topologieaufbau soll daher erst auf dem Network Layer mithilfe
eines Routingprotokolls bewerkstelligt werden. IEEE802.15.4 dagegen zwingt bereits
auf dem Data Link Layer dem Netzwerk eine Topologie auf. Diese Topologien, namentlich
Stern, Peer-to-Peer und Cluster-Tree, belegen aufgrund ihrer unterschiedlichen
Eigenschaften das Routingprotokoll RPL mit Beschr"ankungen bei dem Topologieaufbau
und damit insgesamt schlechteren Netzwerkleistungen.

Daher wird in dieser Arbeit ein L"osungsansatz mithilfe eines TSCH Scheduling Algorithmus
zur Behebung dieses Problems vorgestellt.

% ------------------------------------------------
% Fragestellung und Ziele
% ------------------------------------------------
\subsection{Ziele}
\label{subsec:einleitung_ziele}

% ------------------------------------------------
% Grober Loesungsansatz
% ------------------------------------------------
\subsection{L"osungsansatz}
\label{subsec:einleitung_loesungsansatz}

% --------------------------------------

% ------------------------------------------------
% Grundlagen
% ------------------------------------------------
\section{Grundlagen}
\label{sec:Grundlagen}
% ------------------------------------------------
% TSCH
% ------------------------------------------------
\subsection{IEEE 802.15.4e Timeslotted Channel Hopping}
\label{subsec:grundlagen_tsch}

\subsubsection{Time Divison Multiple Access TDMA}
\label{subsubsec:grundlagen_tdma}

\subsubsection{Enhanced Beacons und Information Elements}
\label{subsubsec:grundlagen_tsch_eb_ie}

\subsubsection{Network Formation Prozess}
\label{subsubsec:grundlagen_tsch_networkformation}
% ------------------------------------------------
% RPL
% ------------------------------------------------
\subsection{Routing Protocol for Low-Power and Lossy Networks RPL}
\label{subsec:grundlagen_rpl}

\subsubsection{Kontrollnachrichten}
\label{subsubsection:grundlagen_rpl_kontrollnachrichten}

DIO

\begin{lstlisting}[frame=single]
0                   1                   2                   3
0 1 2 3 4 5 6 7 8 9 0 1 2 3 4 5 6 7 8 9 0 1 2 3 4 5 6 7 8 9 0 1
+-+-+-+-+-+-+-+-+-+-+-+-+-+-+-+-+-+-+-+-+-+-+-+-+-+-+-+-+-+-+-+-+
| RPLInstanceID |Version Number |             Rank              |
+-+-+-+-+-+-+-+-+-+-+-+-+-+-+-+-+-+-+-+-+-+-+-+-+-+-+-+-+-+-+-+-+
|G|0| MOP | Prf |     DTSN      |     Flags     |   Reserved    |
+-+-+-+-+-+-+-+-+-+-+-+-+-+-+-+-+-+-+-+-+-+-+-+-+-+-+-+-+-+-+-+-+
|                                                               |
+                                                               +
|                                                               |
+                            DODAGID                            +
|                                                               |
+                                                               +
|                                                               |
+-+-+-+-+-+-+-+-+-+-+-+-+-+-+-+-+-+-+-+-+-+-+-+-+-+-+-+-+-+-+-+-+
|   Option(s)...
+-+-+-+-+-+-+-+-+
\end{lstlisting}

DAO

DIS

\subsubsection{Network Formation Prozess}
\label{subsubsec:grundlagen_rpl_networkformation}


% ------------------------------------------------
% Contik/Cooja
% ------------------------------------------------
\subsection{Contiki}
\label{subsec:grundlagen_contiki}

\subsubsection{Cooja}
\label{subsubsec:grundlagen_cooja}


% ------------------------------------------------
% Relevante Arbeiten
% ------------------------------------------------
\section{Relevante Arbeiten}
\label{sec:relevante_arbeiten}

\subsection{The Love-Hate Relationshiop between IEEE802.15.4 and RPL}
\label{subsec:ra_love-hate}

\subsection{Orchestra: Robust Mesh Networks Through Autonomously Scheduled TSCH}
\label{subsec:ra_orchestra}

\subsection{DeTAS: Decentralized Traffic Aware Scheduling in 6TiSCH Networks:
Design and Experimental Evaluation}
\label{subsec:ra_detas}

\subsection{Minimal 6TiSCH Configuration}
\label{subsec:ra_minimal_6tisch}


% ------------------------------------------------
% Design
% ------------------------------------------------
\section{Design und Implementierung}
\label{sec:design_implementierung}

% ------------------------------------------------
% Design
% ------------------------------------------------


\subsection{Gr"osse von Kontrollnachrichten}
\label{subsec:di_ctrlmsg_length}

Im Rahmen der Entwicklung des Algorithmus kam die urspr"ungliche Idee auf,
die Kontrollnachrichten von LR-WPAN und RPL innerhalb eines Paketes zu senden,
weshalb die Gr"osse von Enhanced Beacons und DODAG Information Objects analysiert
wurden.

% ------------------------------------------------

\subsubsection{Enhanced Beacon}
\label{subsubsec:di_eb}
Nach Minimal-6TiSCH beinhaltet ein minimales Enhanced Beacon folgende Teile:

Aufbau Enhanced Beacon
\begin{enumerate}
  \item Header IE Header
  \item Payload IE Header
  \item MLME-SubIE TSCH Synchronization
  \item MLME-SubIE TSCH Timeslot
  \item MLME-SubIE Ch. Hopping
  \item MLME-SubIE TSCH Slotframe and Link
\end{enumerate}

% ------------------------------------------------
\subsubsection{DODAG Information Objects}
\label{subsubsec:di_dio}




% ------------------------------------------------
\subsection{L"osungsansatz}
\label{subsec:di_loesungsansatz}


Als L"osungsansatz werden wir eine strikte Trennung von Kontrollnachrichten und normalen
Datentransfer vornehmen und im weiteren nur die Kontrollnachrichten betrachten.
Diese werden nach der Idee von Orchestra in Trafficplanes mit eigener Slotframestruktur
eingeteilt. Dadurch kann f"ur jede Kontrollnachricht in ihrer eigenen Trafficplane
ein privater angepasster Schedule angewendet werden. Diese lokalen Trafficplanes
werden anschliessend global auf die Slotframestruktur abgebildet, indem f"ur jede
Kontrollnachricht eigene wiederkehrende Timeslots innerhalb des Slotframes festgelegt werden.
Dadurch ergeben sich zwei Punkte, welche betrachtet werden m"ussen:
\begin{enumerate}
  \item der lokale Schedule zur Abarbeitung der Kontrollnachrichten durch Slotframestruktur
  \item ein Algorithmus zur Einteilung des Timeslot- und Channel-Offset anhand der Trafficplanes.
\end{enumerate}

Dieser Algorithmus soll dabei m"oglichst autonom in abh"angigkeit zur Topologie
die Erstellung der Slotframestruktur organisieren. Daf"ur wird beim Scheduling
des Timeslot-Offsets ein Konzept aus DeTAS verwendet, indem zwischen den Trafficplanes
gewechselt wird (bsp. mit 3 Trafficplanes Enhanced Beacons EB, DODAG Information Object DIO,
Unicast RPL Nachrichten DOA, DIS: EB -> DIO -> DOA/DIS -> EB -> DIO -> ...)
Eine solche einmalige Zusammenfassung der Kontrollnachrichten bezeichnen wir
fortan als Controlmessagestack CMS.

% ------------------------------------------------
\subsubsection{Eigenschaften}
\label{subsubsec:di_eigenschaften}


\begin{itemize}
  \item dezentraler, dynamisches Scheduling der Kontrollnachrichten
  \item Austausch der Informationen durch zus"atzliche Steuerungsmeldungen
  eingebettet in den Kontrollnachrichten
  \item Trennung der Kontrollnachrichten zum Datentransfer
  \item Dedizierter Bereich pro Technologie und Kontrollnachricht (Enhanced Beacons, DODAG Information Object,
  DODAG Information Solicitation / Destination Advertisement Object)
\end{itemize}

-dezentraler Ansatz durch Versenden zus"atzlicher Steuerungsmeldungen
welche in den Kontrollnachrichten der Technologien eingebettet sind.
IEEE 802.15.4e TSCh -> EB
RPL -> DIO
RPL -> DIS/DAO

-trennung kontrollnachrichten und datentransfer
-eigener plane pro kontrollnachricht (eb, dio, dao/dis)
-konfliktfreier schedule fuer kontrollnachrichten

 Damit wird der globale Schedule von TSCH
-feste Frequenz innerhalb eines slotframes
-Timeslot Bitmap zur Angabe des lokalen Schedules
-selbstst"andige Konflikte




\begin{itemize}
  \item Gr"osse des EB Bereichs N\_EB
    \begin{itemize}
      \item N\_EB hat in allen Bereichen des Netzwerks als globale Gr"osse
      den gleichen Wert. Der Wert entspricht dabei der h"ochsten Netzwerkdichte + 1
      \item N\_EB hat einen variablen Wert und ist ab"hangig von der Netzwerkdichte
      in dem jeweiligen Teilbereichen des Netzwerkes.
    \end{itemize}
  \item Auswahl der Frequenz
    \begin{itemize}
      \item F"ur die Kontrollnachrichten wird pro Slotframe eine feste Frequenz vorgegeben
      und nur die beteiligten Links k"onnen f"ur die Kontrollnachrichten verwendet werden.
      Dadurch sind die Planes
    \end{itemize}

  \item Linkauswahl
    \begin{itemize}
      \item Die Linkauswahl wird nach dem Zufallsprinzip ausgew"ahlt.
      \item Der erste freie Link wird verwendet.
    \end{itemize}

\end{itemize}

% ------------------------------------------------
\subsubsection{Ablaufplan}
\label{subsubsec:di_ablaufplan}

\begin{description}
  % Netzwerkaufbau
  \item [Netzwerkaufbau] Ablauf beim ersten Aufbau eines Netzwerkes und den
  ben"otigten Schritten zus"atzlich zum normalen Prozess
  \begin{enumerate}
    \item Knoten schaltet auf TSCH Mode ein
    \item Knoten initialisiert Network Formation Data Plane anhand Standard
    \item Knoten sendet Enhanced Beacon mit TSCH Network Formation Schedule Information Element
  \end{enumerate}

  % Netzwerkbeitritt
  \item [Netzwerkbeitritt] Prozess eines Knoten, welcher einem bestehenden
  PAN beitretet
  \begin{enumerate}
    \item Knoten empf"angt EB
    \item Knoten h"ort nach Regel [2] und [3] \ref{subsubsec:di_regeln}
    \item Analyse aller TSCH Network Formation Schedule IE aus erhaltenen EBs
    \begin{enumerate}
      \item Lege Kontrollnachrichtenfrequenz mit momentan verwendeter Frequenz fest
      \item Berechne weitere Links mit gleicher Kontrollnachrichtenfrequenz
      innerhalb des vorgegeben EB Bereich N\_EB
    \end{enumerate}
    \item Setze Links in TS Bitmap anhand TSCH Network Formation Schedule IE
    \item W"ahle einen freien Link nach dem Zufallsprinzip als Sendeslot aus
    \item Erweitere N\_EB um 1
    \item Belege eigene TS Bitmap anhand Regel [4] \ref{subsubsec:di_regeln}
    \item Empfange Links innerhalb N\_EB
    \item Sende eigenes Enhanced Beacon im n"achsten Slotframe und belegten Timeslot
  \end{enumerate}
\end{description}

% ------------------------------------------------
\subsubsection{Kollisionsbehandlung}
\label{subsubsec:di_kollisionsbehandlung}

Im Rahmen des Schedules k"onnen zwei Arten der Kollision festgestellt werden,
einmal die Kollision unterhalb und die Kollision oberhalb. Die Kollision unterhalb
tritt haupts"achlich im Rahmen des Network Formation Prozess auf, wenn zwei Kindknoten
gleichzeitig sich an einen Parentknoten binden wollen. Die Kollision oberhalb
erscheint, wenn ein Kindknoten sich an zwei Parentknoten binden m"ochte. Dieser
Kollisionstyp erscheint nur, wenn der Bereich der Kontrolnachrichten anhand der
lokalen Netzwerkdichte ermittelt wird.

\begin{description}

  \item [Kollision unterhalb] Hierbei erkennt ein Parentknoten, dass zwei oder
  mehr Kindknoten den gleichen Timeslot verwenden wollen, dabei reagieren sowohl
  Parentknoten und Kindknoten unterschiedlich.
  \begin{lstlisting}[frame=single]
  // Platzhalter
  //     n0
  //    /  \
  //  n1    n2
  //
  // Bitmaps fuer n0
  // 1.SF: 10
  // 2.SF: 1C
  // 3.SF: 1000
  // 4.SF: 1110 (n1 und n2 haben zufaellig Timeslot 2 und 3 ausgewaehlt)
  \end{lstlisting}
  %
  \begin{description}
    % Parentknoten
    \item [Aktionen durch Parentknoten]
    %
    \begin{enumerate}
      \item Parentknoten empf"angt innerhalb eines freien Timeslot mehrere Enhanced Beacons
      und erkennt damit eine Kollision
      \item erweitere N\_EB um 2
      \item belege Timeslot Bitmap anhand Regel [4] \ref{subsubsec:di_regeln} (Timeslot wird nicht markiert)
      \item setze Pending Flag
      \item sende eigenes Enhanced Beacon zum n"achsten Zeitpunkt
    \end{enumerate}
    %
\begin{lstlisting}[frame=single]
                    1                   2                   3
0 1 2 3 4 5 6 7 8 9 0 1 2 3 4 5 6 7 8 9 0 1 2 3 4 5 6 7 8 9 0 1
+-+-+-+-+-+-+-+-+-+-+-+-+-+-+-+-+-+-+-+-+-+-+-+-+-+-+-+-+-+-+-+-+
|ID = 0x1a|1|0|0| Len = 4       |1 0 0 0 0 0 0 0|     ...
+-+-+-+-+-+-+-+-+-+-+-+-+-+-+-+-+-+-+-+-+-+-+-+-+-+-+-+-+-+-+-+-+
\end{lstlisting}

    % Kindknoten
    \item [Aktionen durch Kindknoten] Der Kindknoten empf"angt weitere
    Enhanced Beacons von Parentknoten
    %
    \begin{enumerate}
      \item Knoten empf"angt Enhanced Beacon von Parentknoten
      \item Analyse TSCH Network Formation Schedule Information Element
      %
      \begin{enumerate}
        \item Betrachte Timeslot Bitmap
        %
        \begin{enumerate}
          \item Wenn eigener Timeslot markiert ist war die Kommunikation erfolgreich
          und der Link wird behalten
          \item Wenn eigener Timeslot nicht markiert ist, fand eine Kollision mit
          fremden Knoten statt
          %
          \begin{enumerate}
            \item Setze Pending Flag
            \item W"ahle einen freien Link nach dem Zufallsprinzip aus
          \end{enumerate}
          %
          \item Wenn kein weiterer freier Timeslot zum Beitreten eines Knoten
          vorhanden ist, wird N\_EB um 1 erweitert
          \item Setze Timeslot Bitmap anhand Regel [4] \ref{subsubsec:di_regeln}
        \end{enumerate}
        %
      \end{enumerate}
      %
    \item Sende eigenes Enhanced beacon zum n"chstm"oglichen Zeitpunkt
  \end{enumerate}
  %
  \end{description}
  %
  % Kollision oberhalb
  \item [Kollision oberhalb] Ein Kindknoten will eine Verbindung mit zwei Parentknoten
  erstellen, erkennt aber anhand den angebotenen Timeslots, dass immer eine Kollision
  mit dem jeweils anderen Parentknoten auftreten wird.

  \begin{lstlisting}[frame=single]
  // Platzhalter
  //     n0
  //    /  \
  //  n1    n2
  //    \  /
  //     n3
  //
  // Bitmaps
  // n0: 1110
  // n1: 110
  // n2: 101
  // n1 und n2 im weiteren Parentknoten PK, n3 Kindknoten KK
  \end{lstlisting}
  \begin{description}
  \item [Aktion des Kindknoten]
    \begin{enumerate}
      \item Kindknoten empf"angt Enhanced Beacons von mehreren Parentknoten
      \item Analyse TSCH Network Formation Schedule Information Element
      \begin{enumerate}
        \item Betrachte Timeslot Bitmap
        \item Initialisiere interne Bitmap anhand den Werten des ersten Parentknoten
        \item Vergleiche Werte von anderen Parentknoten f"ur jeden Timeslot
        \begin{enumerate}
          \item Wenn interner Bitmap den Wert 1 anzeigt und der neue Wert ebenfalls
          eine 1 anzeigt "'andert sich der Wert nicht.
          \item Wenn interne Bitmap den Wert 1 anzeigt und der neue Wert den Wert 0
          aufweist, wird der Timeslot gesondert markiert.
          \item Wenn interne Bitmap den Wert 0 aufweist und der neue Wert den Wert 1,
          wird der Timeslot gesondert markiert.
          \item Wenn interne Bitmap den Wert 0 hat und der neue Wert ebenfalls
          den Wert 0, tritt keine "'Anderung auf.
        \end{enumerate}
        \item Wenn der Wert 0 vorhanden, findet keine Kollision oberhalb statt,
        weshalb der Timeslot ausgew"ahlt werden kann und eigenes Enhanced Beacon
        mit Bitmap gesendet werden kann
        \item Wenn der Wert 0 nicht vorhanden, werden an alle gesondert markierten
        Timeslots ein Enhanced Beacon mit aktiven Pending Flag und Bitmap gesendet.
      \end{enumerate}
    \end{enumerate}
  \item [Aktion der Parentknoten]
    \begin{enumerate}
      \item Parentknoten erh"aelt Enhanced Beacon von Kindknoten mit Pending Flag
      \item Parentknoten erweitert N\_EB um 1
      \item Parentknoten sendet Enhanced Beacon zum n"achstm"oglichen Zeitpunkt
    \end{enumerate}
  \end{description}
\end{description}


% ------------------------------------------------
\subsubsection{Regeln}
\label{subsubsec:di_regeln}
Im Rahmen des Algorithmus m"ussen die Knoten einige Regeln beachten.

\begin{enumerate}
  \item Die Kommunikation der Kontrollnachrichten wird innerhalb eines
  Slotframes nur auf einer festen Frequenz erfolgen, womit eine begrenzte Anzahl
  an m"oglichen Links festgelegt werden. Nach dem Slotframe wird die Frequenz
  durch den normalen Channel Hopping Prozess ge"andert. Diese Frequenz wird vortan
  als Kontrollnachrichtenfrequenz.
  \item Vor der Aktivierung des TSCH Mode muss ein Knoten zwei Slotframes
  auf weitere Enhanced Beacons warten um weitere potentzielle Parentknoten
  empfangen zu k"onnen.
  \item Ein Knoten h"ort weitere Slotframes auf Enhanced Beacons, solange ein
  Parentknoten zwei aufeinanderfolgende Enhanced Beacons mit inaktivem Pending Flag
  sendet.
  \item Ein Feld auf der Timeslot Bitmap, innerhalb des TSCH Network Formation Schedule IE,
  wird markiert, wenn der Knoten selber oder ein direkter Nachbar den Timeslot
  zum Senden seines Enhanced Beacon verwendet. Das Feld wird nicht markiert,
  wenn dementsprechend der Knoten oder ein direkter Nachbar nicht auf dem Timeslot
  sendet oder der Knoten eine Kollision auf dem Timeslot erkannt hat.
\end{enumerate}

% ------------------------------------------------
\subsubsection{TSCH Network Formation Schedule Information Element}
\label{subsubsec:di_nfsie}

Eigenes Information Element

\begin{lstlisting}[frame=single]

                    1                   2                   3
0 1 2 3 4 5 6 7 8 9 0 1 2 3 4 5 6 7 8 9 0 1 2 3 4 5 6 7 8 9 0 1
+-+-+-+-+-+-+-+-+-+-+-+-+-+-+-+-+-+-+-+-+-+-+-+-+-+-+-+-+-+-+-+-+
|Sched-Id |P|T|C| Length TSBM   |   TS Bitmap   | Channel Bitmap
+-+-+-+-+-+-+-+-+-+-+-+-+-+-+-+-+-+-+-+-+-+-+-+-+-+-+-+-+-+-+-+-+
| (opt.)        |
+-+-+-+-+-+-+-+-+
\end{lstlisting}

\begin{description}
  \item [Schedule-ID] (5 Bit) Nummer (0x00 - 0x1f) welche einem Kontrollplane eindeutig
  zugeordnet wird und diesen einleitet
    \begin{center}
      \begin{tabular}{ |l|l| }
        \hline
        Bereich & Beschreibung \\ \hline
        0x00 - 0x19 & nicht verwendet \\ \hline
        0x1a & Enhanced Beacons \\ \hline
        0x1b & DIO-Nachrichten \\ \hline
        0x1c & DIO/DIS-Nachrichten \\ \hline
        0x1d - 0x1f & nicht verwendet \\
        \hline
      \end{tabular}
    \end{center}
  \item [Pending Flag] (1 Bit) ein aktives Pending Flag zeigt an, dass der Schedule momentan
  nicht fest zugeordnet ist und noch Probleme ausstehen (Kollision)
  \item [Type Flag] (1 Bit) Beschreibt die L"ange der TS Bitmap
    \begin{center}
      \begin{tabular}{|l|l|l|l|}
        \hline
        Flag & Typ & Beschreibung & M"ogliche Timeslots \\ \hline
        0 & Short Length & L"ange der TS Bitmap sind 8 Bit & 256 \\ \hline
        1 & Long Length & L"ange der TS Bitmap sind 16 Bit & 65536 \\
        \hline
      \end{tabular}
    \end{center}
  \item [Channel Flag] (1 Bit) liefert den Hinweis "'uber eine aktive Channel Bitmap
  \item [Length TSCH Network Formation Schedule Information Element] (1/2 Byte)
  beschreibt die Gr"osse der Timeslot Bitmap
  \item [Timeslot Bitmap] (variable vielfache von 8 Bit) Liefert die Belegung des lokalen
  Schedules
  \item [Channel Bitmap] (16 Bit) optionale Bitmap, zur Angabe der aktiven Channel Offsets
\end{description}


% ------------------------------------------------
% Implementierung
% ------------------------------------------------
% ------------------------------------------------
% Implementierung
% ------------------------------------------------

\subsection{Implementierung}
\label{subsec:di_implementierung}

% ------------------------------------------------
\subsubsection{Pseudocode}
\label{subsubsec:di_pseudocode}

Initialisierung
\begin{lstlisting}[frame=single]
Platzhalter
\end{lstlisting}

Knoten beitreten
\begin{lstlisting}[frame=single]
Platzhalter
\end{lstlisting}

Kollisionsbehandlung unterhalb
\begin{lstlisting}[frame=single]
Platzhalter
\end{lstlisting}

...
\begin{lstlisting}[frame=single]
Platzhalter
\end{lstlisting}


% ------------------------------------------------
\subsubsection{Implementierung in Cooja}
\label{subsubsec:di_impl_cooja}
c cooja

beachte datenstrukturen


% ------------------------------------------------
% Analyse und Evaluation
% ------------------------------------------------
\section{Analyse und Evaluation}
\label{sec:evaluation}

% ------------------------------------------------
% Prozess
% ------------------------------------------------

\subsection{Testprozess in Cooja/Contiki}
\label{subsec:ae_testprozess}

Der Testablauf wird zwei Bereiche aufgeteilt, den

\begin{itemize}
  \item Netzwerkaufbau
  \item Beitrittsprozess
\end{itemize}

\subsection{Topologien}
\label{subsec:ae_topologien}

F"ur Testzwecke wurden zwei Netzwerktopologien ausgew"ahlt um den Algorithmus mit
mehreren Anwendungsf"allen zu kontrollieren

Netzwerk A
Platzhalter - Durch Bild ersetzen
\begin{lstlisting}[frame=single]
//
// n0 - n1 - n2
//  |    |    |
// n3 - n4 - n5
//  |    |    |
// n6 - n7 - n8
//
\end{lstlisting}

Netzwerk B
Knoten n6 betritt das bereits etablierte Netzwerk mit zwei Parentknoten
Platzhalter - Durch Bild ersetzen
\begin{lstlisting}[frame=single]
//
//          n0
//         /  \
//      n1     n2
//   / | | \  /
// n3 n4 n5 n6
//
\end{lstlisting}

\subsection{Leistungsmetriken}
\label{subsec:ae_leistungsmetriken}


\begin{description}
  \item [Network Formation Time] Zeit vom ersten Enhanced Beacon, bis letzter
  Knoten dem Netzwerk beigetreten ist
  \item [Joining Time] Zeit die ein Knoten ben"otigt um einem Netzwerk beizutreten
  (abh"angig von festgelegten Regeln)
  \item [Overhead] Welchen Overhead an Nachrichten erzeugt der Algorithmus
  \item [Zeitraum Kontrollnachrichten] Wieviel Zeit wird f"ur die Kontrollnachrichten
  reserviert
  \item [Verhalten Algorithmus "'uber l"angeren Zeitraum] Wie verhalten sich die Leistungsmetriken
  "'uber einen l"angeren Zeitraum?
  \item [Aktivit"atszeit Knoten] Wie lange ist ein Knoten im Rahmen des Schedules aktiv
  im Verh"altnis zur tats"achlichen Aktivit"at.
  \item [Energieverbrauch] Wieviel Energie
\end{description}

% ------------------------------------------------
% Referenzwerte Minimal 6TiSCH
% ------------------------------------------------
\subsection{Network Formation mit Minimal 6TiSCH Configuration}
\label{subsec:ae_6tisch}

% ------------------------------------------------
% Ergebnisse Algorithmus
% ------------------------------------------------
\subsection{Network Formation mit Algorithmus}
\label{subsec:ae_algorithmus}

% ------------------------------------------------
% Vergleich und Evaluation
% ------------------------------------------------
\subsection{Bewertung der Ergebnisse}
\label{subsec:ae_bewertung}


% ------------------------------------------------
% Zusammenfassung
% ------------------------------------------------
\section{Zusammenfassung}
\label{sec:zusammenfassung}

% ------------------------------------------------
% Schlussfolgerungen aus den Ergebnisse
% ------------------------------------------------

\subsection{Schlussfolgerungen}
\label{subsec:z_schlussfolgerungen}

% ------------------------------------------------
% Meinung zu Ergebnisse im Vergleich zu Referenzwerten
% aufgrund logischer Schlussfolgerung
% ------------------------------------------------
\subsection{Diskussion}
\label{subsec:z_diskussion}

% ------------------------------------------------
% Verweis auf Erweiterungen der Arbeit und neue Problemstellungen
% ------------------------------------------------
\subsection{Weiterfuehrende Arbeiten}
\label{subsec:z_weiterfuehrende_arbeiten}


% ------------------------------------------------



% ------------------------------------------------
% .... Literaturverzeichnis
% -------------------------
\printbibliography
% ------------------------------------------------

\end{document}
